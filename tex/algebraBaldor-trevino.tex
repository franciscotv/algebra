\documentclass[12pt]{article}
\usepackage{amsmath}
\usepackage[utf8]{inputenc}
\usepackage[spanish]{babel}
\usepackage{enumitem}

\title{Álgebra de Baldor}
\author{Francisco Treviño zerofrancisco@gmail.com}

\begin{document}
\maketitle

\section*{Ejercicio 89}
Factorar o descomponer en dos factores.\\
Para comprobar que la respuesta es correcta, multiplica los factores del resultado, reacomoda y verifica que tienes la expresión dada en el ejercicio.

\begin{enumerate}[label=\bfseries Ejercicio \arabic*:]
  \item $$a^2 + ab$$
El elemento común a ambos monomios es $a$, y lo utilizamos como factor común lo cual nos da:
$$a(a + b)$$
  \item $$b + b^2$$
El elemento común a ambos monomios es $b$, y lo utilizamos como factor común:
$$b(1+b)$$
  \item $$x^2 + x$$
El elemento común a ambos monomios es $x$, lo utilizamos como factor común, lo cual nos da:
$$x(x+1)$$
  \item $$3a^3 - a^2$$
El elemento común a ambos monomios es $a^2$, al factorizarlo obtenemos:
$$a^2(3a - 1)$$
  \item $$x^3 - 4x^4$$
El elemento común a ambos monomios es $x^3$ el cual factorizamos obteniendo:
$$x^3(1 - 4x)$$
  \item $$5m^2 + 15m^3$$
Factorizamos 5 como máximo común divisor y $m^2$ también como factor común:
$$5m^2(1 + 3m)$$
  \item $$ab - bc$$
El elemento común en ambos monomios es $b$, factorizando tenemos:
$$b(a-c)$$
  \item $$x^2y + x^2z$$
El elemento común a ambos monomios es $x^2$, lo factorizamos para obtener:
$$x^2(y + z)$$
  \item $$2a^2x + 6ax^2$$
El máximo común divisor de ambos monomios es 2 y factorizamos $ax$ porque pertenece a ambos monomios también:
$$2ax(a + 3x)$$
  \item $$8m^2 - 12mn$$
Factorizamos 4 como máximo común divisor de ambos monomios y tomamos $m$ como factor común también:
$$4m(2m - 3n)$$
  \item $$9a^3x^2 - 18ax^2$$
El máximo común divisor de ambos monomios es 9, además $ax^2$ es el monomio presente en ambos, así que lo factorizamos:
$$9ax^2(a^2 -2)$$
  \item $$15c^3d^2 + 60c^2d^3$$
El máximo común divisor es 15 y el monomio presente en ambos monomios es $c^2d^2$, así que factorizando:
$$15c^2d^2(c + 4d)$$
  \item $$35m^2n^3 - 70m^3$$
El máximo común divisor es 35 y el monomio presente en ambos es $m^2$, así que tenemos:
$$35m^2(n^3 - 2m)$$
  \item $$abc + abc^2$$
El monomio que podemos factorizar de ambos monomios originales es $abc$ dado que está presente en ambos:
$$abc(1 + c)$$
  \item $$24a^2xy^2 -36x^2y^4$$
El máximo común divisor es 6 y podemos extraer como monomio común $xy^2$ y obtenemos:
$$6xy^2(4a^2 - 6xy^2)$$
  \item $$a^3 + a^2 + a$$
Podemos factorizar $a$ de todos los monomios obteniendo:
$$a(a^2 + a +1)$$
  \item $$4x^2 - 8x + 2$$
El único factor común entre los tres monomios es 2, factorizando obtenemos:
$$2(2x^2 - 4x + 1)$$
  \item $$15y^3 + 20y^2 - 5y$$
Tenemos como factor común en los tres monomios a $5y$, factorizando obtenemos:
$$5y(3y^2 + 4y - 1)$$
  \item $$a^3 -a^2x + ax^2$$
El único factor común entre los tres monomios es $a$, factorizando obtenemos:
$$a(a^2 - ax + x^2)$$
  \item $$2a^2x +2ax^2 - 3ax$$
El monomio que podemos extraer como elemento común a los originales 3 monomios es $ax$:
$$ax(2a + 2x - 3)$$
  \item $$x^3 +x^5 -x^7$$
Podemos extraer $x^3$ como factor común obteniendo (recuerda que al multiplicar monomios con la misma base, los exponentes se suman):
$$x^3(1 + x^2 - x^4)$$
  \item $$14x^2y^2 -28x^3 + 56x^4$$
Factorizamos 14 como máximo común divisor y tomamos $x^2$ como factor común:
$$14x^2(y^2 - 2x + 4x^2)$$
  \item $$34ax^2 + 51a^2y - 68ay^2$$
No encontramos común divisor y el único factor común entre los monomios originales es $a$:
$$a(34x^2 + 51ay - 68y^2)$$
  \item $$96 - 48mn^2 + 144n^3$$
No tenemos factor común, pero podemos factorizar el máximo común divisor:
$$4(24 - 12mn^2 + 36n^3)$$
  \item $$a^2b^2c^2 - a^2c^2x^2 + a^2c^2y^2$$
Podemos abstraer como factor común $a^2c^2$ dado que se encuentra en los tres monomios originales:
$$a^2c^2(b^2 - x^2 + y^2)$$
  \item $$55m^2n^3x + 110m^2n^3x^2 - 220m^2y^3$$
Tomamos 55 como máximo común divisor, y $m^2$ como factor común entre los monomios originales:
$$55m^2(n^3x + 2 n^3x^2 - 4y^3)$$
  \item $$93a^3x^2y - 62a^2x^3y^2 - 124a^2x$$
Tomamos 31 como máximo común divisor, luego podemos ver que el factor común entre los monomios originales es $a^2x$, factorizando estos dos tenemos:
$$31a^2x(3axy - 2x^2y^2 - 4)$$
  \item $$x - x^2 + x^3 - x^4$$
El único factor común entre los monomios originales es $x$:
$$x(1 - x + x^2 - x^3)$$
  \item $$a^6 - 3a^4 + 8a^3 - 4a^2$$
El único factor común entre los monomios originales es $a^2$:
$$a^2(a^4 - 3a^2 + 8a - 4)$$
  \item $$25x^7 - 10x^5 + 15x^3 - 5x^2$$
El máximo común divisor es 5 y encontramos $x^2$ en todos los monomios, factorizando:
$$5x^2(5x^5 -2x^3 + 3x -1)$$
  \item $$x^{15} - x^{12} + 2x^9 - 3x^6$$
El término común a los monomios es $x^6$, factorizamos recordando que al multiplicar monomios de la misma base los exponentes se suman:
$$x^6(x^9 - x^6 +2x^3 - 3)$$
  \item $$9a^2 - 12ab + 15a^3b^2 - 24 ab^3$$
El máximo común divisor es 3 y el término común a los monomios es $a$, factorizamos:
$$3a(3a - 4b + 5a^2b^2 - 8b^3)$$
  \item $$16x^3y^2 - 8x^2y -24x^4y^2 - 40x^2y^3$$
El máximo común divisor es 8 y el término común a los monomios originales es $x^2y$, factorizamos:
$$8x^2y(2xy - 1 - 3x^2y - 5y^2)$$
  \item $$12m^2n + 24m^3n^2 - 36m^4n^3 + 48m^5n^4$$
El máximo común divisor es 12 y el término común es $m^2n$, factorizamos:
$$12m^2n(1 + 2mn - 3m^2n^2 + 4m^3n^3)$$
  \item $$100a^2b^3c - 150ab^2c^2 + 50ab^3c^3 - 200abc^2$$
El máximo común divisor entre los monomios es 50 y el término común es $abc$, factorizamos:
$$50abc(2ab^2 - 3bc + b^2c^2 - 4c)$$
  \item $$x^5 - x^4 +x^3 - x^2 +x$$
El único factor común entre los monomios originales es $x$, factorizando:
$$x(x^4 - x^3 + x^2 -x + 1)$$
  \item $$a^2 - 2a^3 + 3a^4 - 4a^5 + 6a^6$$
El único factor común entre los monomios es $a^2$, factorizamos:
$$a^2(1 - 2a + 3a^2 - 4a^3 + 6a^4)$$
  \item $$3a^2b + 6ab - 5a^3b^2 + 8a^2bx + 4ab^2m$$
El máximo común divisor es 1 y el término común a todos los monomios es $ab$, factorizamos:
$$ab(3a + 6 -5a^2b + 8ax + 4bm)$$
  \item $$a^{20} - a^{16} + a^{12} - a^8 + a^4 -a^2$$
El término común a todos los monomios es $a^2$, factorizamos:
$$a^2(a^{18} - a^{14} + a^{10} - a^6 + a^2 -1)$$
\end{enumerate}

\section*{Ejercicio 136}
Simplificar:
\begin{enumerate}[label=\bfseries Ejercicio \arabic*:]
  \item $$\frac{3x}{4y} \times \frac{8y}{9x} \div \frac{z^2}{3x^2}$$
Primero cambiamos la división en multiplicación intercambiando el numerador por denominador y viceversa, luego reducimos factores multiplicando respectivamente:
$$= \frac{3x}{4y} \times \frac{8y}{9x} \times \frac{3x^2}{z^2} = \frac{72 x^3y}{36xyz^2}$$
Luego reducimos factores numéricos y factores comunes:
$$= \frac{2x^2}{z^2}$$
  \item $$\frac{5a}{b} \div \left( \frac{2a}{b^2} \times \frac{5x}{4a^2} \right)$$
Primero efectuamos la multiplicación indicada y luego cambiamos la operación de división por una multiplicación intercambiando numerador por denominador y viceversa. Luego efectuamos las multiplicaciones y reducimos:
$$= \frac{5a}{b} \times \frac{4a^2b^2}{10ax} = \frac{2a^2b}{x}$$
  \item $$\frac{a+1}{a-1} \times \frac{3a - 3}{2a + 2} \div \frac{a^2 + a}{a^2 + a - 2}$$
Primero intercambiamos la división por una multiplicación y factorizamos el trinomio y los demás binomios que nos sea posible:
$$= \frac{a+1}{a-1} \times \frac{3a - 3}{2a + 2} \times \frac{a^2 + a - 2}{a^2 + a}$$
$$= \frac{a+1}{a-1} \times \frac{3(a - 1)}{2(a + 1)} \times \frac{(a + 2)(a - 1)}{a (a + 1)}$$
Reduciendo términos semejantes:
$$= \frac{3(a + 2)(a - 1)}{2a(a + 1)} = \frac{3a^2 + 3a - 6}{2a^2 + 2a}$$
  \item $$\frac{64a^2 - 81b^2}{x^2 - 81} \times \frac{(x - 9)^2}{8a - 9b} \div \frac{8a^2 + 9ab}{(x + 9)^2}$$
Primero convertimos la división en multiplicación invirtiendo numerador y denominador y expandimos las operaciones indicadas y restas de binomios al cuadrado condensados:
$$= \frac{64a^2 - 81b^2}{x^2 - 81} \times \frac{(x - 9)^2}{8a - 9b} \times \frac{(x + 9)^2}{8a^2 + 9ab} = \frac{(8a + 9b)(8a - 9b)}{(x + 9)(x - 9)} \times \frac{(x-9)^2}{8a - 9b} \times \frac{(x + 9)^2}{a(8a + 9b)}$$
Reducimos términos semejantes:
$$= \frac{1}{1} \times \frac{(x - 9)}{1} \times \frac{(x + 9)}{a}$$
Finalmente tenemos, reduciendo la resta de cuadrados:
$$= \frac{(x^2 - 81)}{a}$$
  \item $$\frac{x^2 - x - 12}{x^2 - 49} \times \frac{x^2 - x - 56}{x^2 + x - 20} \div \frac{x^2 - 5x - 24}{x + 5}$$
Intercambiamos la división por una multiplicación intercambiando el numerador por denominador y viceversa y expandimos resta de cuadrados y expandimos trinomios:
$$= \frac{x^2 - x - 12}{x^2 - 49} \times \frac{x^2 - x - 56}{x^2 + x - 20} \times \frac{x + 5}{x^2 - 5x - 24}$$
$$= \frac{(x - 4)(x + 3)}{(x + 7)(x - 7)} \times \frac{(x - 8)(x + 7)}{(x + 5)(x - 4)} \times \frac{x+5}{(x - 8)(x + 3)}$$
Reducimos términos semejantes en numerador y denominador:
$$= \frac{1}{x - 7} \times \frac{1}{1} \times \frac{1}{1} = \frac{1}{x - 7} $$
  \item $$\frac{a^2 - 8a + 7}{a^2 - 11a + 30} \times \frac{a^2 - 36}{a^2 - 1} \div \frac{a^2 - a -42}{a^2 - 4a - 5}$$
Primero intercambiamos la división por una multiplicación y factorizamos los trinomios que nos sea posible, también expandimos las sumas de cuadrados por su par de binomios:
$$= \frac{a^2 - 8a + 7}{a^2 - 11a + 30} \times \frac{a^2 - 36}{a^2 - 1} \times \frac{a^2 - 4a - 5}{a^2 - a -42}$$
$$= \frac{(a - 7)(a - 1)}{(a - 6)(a - 5)} \times \frac{(a + 6)(a - 6)}{(a + 1)(a - 1)} \times \frac{(a - 5)(a + 1)}{(a - 7)(a + 6)}$$
Finalmente reducimos términos semejantes en el numerador y denominador:
$$= \frac{1}{1} \times \frac{1}{1} \times \frac{1}{1} = 1$$
  \item $$\frac{x^4 - 27x}{x^2 + 7x - 30} \times \frac{x^2 + 20x + 100}{x^3 + 3x^2 + 9x} \div \frac{x^2 - 100}{x - 3}$$
Expresamos la división como una división intercambiando el numerador y denominador:
$$= \frac{x^4 - 27x}{x^2 + 7x - 30} \times \frac{x^2 + 20x + 100}{x^3 + 3x^2 + 9x} \times \frac{x - 3}{x^2 - 100}$$
Expandimos la resta de cubos, si no la recuerdas es un buen momento para repasarla:
$$= \frac{x(x^3 - 27)}{x^2 + 7x - 30} \times \frac{x^2 + 20x + 100}{x^3 + 3x^2 + 9x} \times \frac{x - 3}{x^2 - 100}$$
$$= \frac{x(x - 3)(x^2 + 3x + 9)}{x^2 + 7x - 30} \times \frac{x^2 + 20x + 100}{x^3 + 3x^2 + 9x} \times \frac{x - 3}{x^2 - 100}$$
Factorizamos los trinomios y expandimos la resta de cuadrados:
$$= \frac{x(x - 3)(x^2 + 3x + 9)}{(x + 10)(x - 3)} \times \frac{(x + 10)^2}{x(x^2 + 3x + 9)} \times \frac{x - 3}{(x + 10)(x - 10)}$$
Reducimos términos semejantes en el numerador y denominador:
$$= \frac{1}{1} \times \frac{1}{1} \times \frac{x - 3}{(x - 10)} = \frac{x - 3}{x - 10}$$
  \item $$\frac{(a^2 + 1)}{3a -6} \div \left( \frac{a^3 +a}{6a - 12} \times \frac{4x + 8}{x - 3} \right)$$
Primero realizamos la multiplicación indicada y después cambiamos la división por una multiplicación intercambiando el numerador y denominador:
$$= \frac{(a^2 + 1)}{3a -6} \div \left( \frac{(a^3 +a)(4x + 8)}{(6a - 12)(x - 3)} \right)$$
$$= \frac{(a^2 + 1)}{3a -6} \times \frac{(6a - 12)(x - 3)}{(a^3 +a)(4x + 8)}$$
Hacemos un poco de factorización:
$$= \frac{(a^2 + 1)}{3a -6} \times \frac{2(3a - 6)(x - 3)}{a(a^2 +1)4(x + 2)}$$
Reducimos términos semejantes:
$$= \frac{1}{1} \times \frac{(x - 3)}{2a(x + 2)} = \frac{(x - 3)}{2a(x + 2)}  = \frac{x - 3}{2ax + 4a}$$
  \item $$\frac{8x^2 - 10x -3}{6x^2 +13x + 6} \times \frac{4x^2 - 9}{3x^2 + 2x} \div \frac{8x^2 + 14x + 3}{9x^2 + 12x + 4}$$
Comenzamos por invertir el numerador y denominador para convertir la división en multiplicación:
$$= \frac{8x^2 - 10x -3}{6x^2 +13x + 6} \times \frac{4x^2 - 9}{3x^2 + 2x} \times \frac{9x^2 + 12x + 4}{8x^2 + 14x + 3}$$
Ahora factorizamos los trinomios cuadrados multiplicando y dividiendo por el coeficiente del término cuadrado:
$$8x^2 - 10x -3 = (64x^2 - 10(8)x - 24)(1/8) = (8x - 12)(8x + 2)(1/8)$$
$$ = \frac{8x - 12}{4} \times \frac{8x + 2}{2} = (2x - 3)(4x + 1)$$
El siguiente trinomio tiene solución si aplicamos la fórmula general de la ecuación cuadrática: 
\begin{equation}\label{imp}
  6x^2 +13x + 6
\end{equation}
por lo tanto el trinomio puede factorizarse; aplicamos la fórmula general cuadrática:
$$\frac{-b \pm \sqrt{b^2 - 4ac}}{2a} = \frac{-13 \pm \sqrt{13^2 - 4(6)(6)}}{2(6)} = \frac{-13 \pm \sqrt{169 - 144}}{12}$$
$$= \frac{-13 \pm 5}{12}$$
lo cual nos arroja como resultados $-3/2$ y $-2/3$, esto es:
$$(x + 3/2)(x + 2/3)$$
los cuales multiplicamos por 2 y por 3 respectivamente para quitar coeficientes fraccionarios y tenemos como resultado final:
$$(2x + 3)(3x + 2)$$
el cual podemos verificar multiplicando que nos da el trinomio inicial \ref{imp}.
Tomando otro trinomio del tercer factor para factorizarlo:
$$9x^2 + 12x + 4 = (81x^2 + 12(9)x + 36)(1/9) = (9x + 6)(9x + 6)(1/9)$$
$$\frac{9x + 6}{3} \times \frac{9x + 6}{3} = (3x + 2)^2$$
Tomando el último trinomio del tercer factor tenemos, multiplicando y dividiendo por el coeficiente del término cuadrado:
$$8x^2 + 14x + 3 = (64x^2 + 14(8)x + 24)(1/8)$$
$$= (8x + 12)(8x + 2)(1/8) = \frac{8x + 12}{4} \times \frac{8x + 2}{2}$$
$$= (2x + 3)(4x + 1)$$
Retomando la expresión original sustituyendo los nuevos términos que obtuvimos al factorizar los trinomios tenemos:
$$ = \frac{(2x - 3)(4x + 1)}{(2x + 3)(3x + 2)} \times \frac{4x^2 - 9}{3x^2 + 2x} \times \frac{(3x + 2)^2}{(2x + 3)(4x + 1)}$$
Ahora expandimos la resta de cuadrados y usamos factor común:
$$ = \frac{(2x - 3)(4x + 1)}{(2x + 3)(3x + 2)} \times \frac{(2x + 3)(2x -3)}{x(3x + 2)} \times \frac{(3x + 2)^2}{(2x + 3)(4x + 1)}$$
Finalmente estamos listos para reducir términos semejantes:
$$ = \frac{(2x - 3)}{1} \times \frac{(2x -3)}{x} \times \frac{1}{(2x + 3)} = \frac{4x^2 - 12x + 9}{2x^2 + 3x}$$
  \item $$\frac{(a + b)^2 - c^2}{(a - b)^2 - c^2} \times \frac{(a + c)^2 -b^2}{a^2 + ab -ac} \div \frac{a + b+ c}{a^2}$$
Comenzamos por cambiar la división por una multiplicación intercambiando numerador y denominador:
$$= \frac{(a + b)^2 - c^2}{(a - b)^2 - c^2} \times \frac{(a + c)^2 -b^2}{a^2 + ab -ac} \times \frac{a^2}{a + b+ c}$$
Ahora expandimos las sumas de cuadrados y factorizamos términos comunes:
$$= \frac{[(a+ b) + c][(a+ b) - c]}{[(a - b) + c][(a - b) - c]} \times \frac{[(a + c) + b][(a + c) - b]}{a(a + b - c)} \times \frac{a^2}{a + b+ c}$$
Reducimos términos semejantes:
$$= \frac{1}{[(a - b) - c]} \times \frac{[(a + c) + b]}{1} \times \frac{a}{1}$$
Finalmente tenemos:
$$= \frac{a(a + c + b)}{a - b - c} = \frac{a^2 + ac + ab}{a - b - c}$$
  \item  $$\frac{a^2 - 5a}{b + b^2} \div \left( \frac{a^2 + 6a - 55}{b^2 - 1} \times \frac{ax + 3a}{ab^2 + 11b^2} \right)$$
Empezamos por factorizar el trinomio, expandir la resta de cuadrados y factorizar términos comunes en los dos factores del divisor:
$$= \frac{a^2 - 5a}{b + b^2} \div \left( \frac{(a + 11)(a - 5)}{(b+1)(b - 1)} \times \frac{a(x + 3)}{b^2(a + 11)} \right)$$
ahora podemos cambiar numerador y denominador y hacerlo una multiplicación:
$$= \frac{a^2 - 5a}{b + b^2} \times \left( \frac{(b+1)(b - 1)}{(a + 11)(a - 5)} \times \frac{b^2(a + 11)}{a(x + 3)} \right)$$
Factorizamos términos comunes en el primer factor:
$$= \frac{a(a - 5)}{b(1 + b)} \times \left( \frac{(b+1)(b - 1)}{(a + 11)(a - 5)} \times \frac{b^2(a + 11)}{a(x + 3)} \right)$$
Reducimos términos semejantes:
$$= \frac{1}{1} \times \left( \frac{(b - 1)}{1} \times \frac{b}{(x + 3)} \right) = \frac{b^2 - b}{x + 3}$$
  \item $$\frac{m^3 + 6m^2n + 9mn^2}{2m^2n + 7 mn^2 + 3n^3} \times \frac{4m^2 - n^2}{8m^2 - 2mn -n^2} \div \frac{m^3 + 27n^3}{16m^2 + 8mn +n^2}$$
Iniciamos cambiando numerador por denominador para hacer la división una multiplicación
$$= \frac{m^3 + 6m^2n + 9mn^2}{2m^2n + 7 mn^2 + 3n^3} \times \frac{4m^2 - n^2}{8m^2 - 2mn -n^2} \times \frac{16m^2 + 8mn +n^2}{m^3 + 27n^3}$$
Continuamos factorizando términos comunes y expandiendo la resta de cuadrados y también la suma de cubos:
$$= \frac{m(m^2 + 6mn + 9n^2)}{n(2m^2 + 7 mn + 3n^2)} \times \frac{(2m + n)(2m - n)}{8m^2 - 2mn -n^2} \times \frac{16m^2 + 8mn +n^2}{(m + 3n)(m^2 - 3mn + 9n^2)}$$
Factorizando trinomios cuadrados perfectos:
$$= \frac{m(m + 3n)^2}{n(2m^2 + 7mn + 3n^2)} \times \frac{(2m + n)(2m - n)}{8m^2 - 2mn - n^2} \times \frac{(4m + n)^2}{(m + 3n)(m^2 - 3mn + 9n^2)}$$
Ahora factorizamos los trinomios cuadrados que involucran más álgebra:
$$2m^2 + 7mn + 3n^2$$
Multiplicamos y dividimos por el coeficiente del factor del término cuadrático:
$$= (4m^2 + 7(2)mn + 6n^2)(1/2) = (2m + 6n)(2m + n)(1/2)$$
$$= (m + 3n)(2m + n)$$
Ahora hacemos el mismo procedimiento con el siguiente trinomio cuadrado:
$$8m^2 - 2mn - n^2$$
Comenzamos por multiplicar y dividir por el coeficiente del factor del término cuadrático:
$$64m^2 - 2(8)mn - 8n^2 = (8m - 4n)(8m + 2n)(1/8)$$
$$\frac{(8m - 4n)}{4} \frac{(8m + 2n)}{2} = (2m - n)(4m + n)$$
Sustituyendo los trinomios factorizados en la expresión original:
$$= \frac{m(m + 3n)^2}{n(m + 3n)(2m + n)} \times \frac{(2m + n)(2m - n)}{(2m - n)(4m + n)} \times \frac{(4m + n)^2}{(m + 3n)(m^2 - 3mn + 9n^2)}$$
Reducimos términos semejantes:
$$= \frac{m}{n} \times \frac{1}{1} \times \frac{(4m + n)}{(m^2 - 3mn + 9n^2)} = \frac{m(4m + n)}{n(m^2 - 3mn + 9n^2)}$$
Finalmente:
$$= \frac{4m^2 + mn}{m^2n - 3mn^2 + 9n^3}$$
  \item $$\frac{(a^2 - ax)^2}{a^2 + x^2} \times \frac{1}{a^3 + a^2x} \div \left( \frac{a^3 - a^2x}{a^2 + 2ax + x^2} \times \frac{a^2 - x^2}{a^3 + ax^2} \right)$$
Comenzamos por cambiar la división por una multiplicación invirtiendo el numerador y denominadores:
$$= \frac{(a^2 - ax)^2}{a^2 + x^2} \times \frac{1}{a^3 + a^2x} \times \left( \frac{a^2 + 2ax + x^2}{a^3 - a^2x} \times \frac{a^3 + ax^2}{a^2 - x^2} \right)$$
Continuamos por factorizar el trinomio cuadrado perfecto y expandir la resta de cuadrados, también factorizamos términos comumes:
$$= \frac{(a^2 - ax)^2}{a^2 + x^2} \times \frac{1}{a^2(a + x)} \times \frac{(a + x)^2}{a(a^2 - ax)} \times \frac{a(a^2 + x^2)}{(a + x)(a - x)}$$
Reducimos términos semejantes:
$$= \frac{(a^2 - ax)}{1} \times \frac{1}{a^2} \times \frac{1}{1} \times \frac{1}{(a - x)}$$
Seguimos factorizando:
$$= \frac{a(a - x)}{1} \times \frac{1}{a^2} \times \frac{1}{1} \times \frac{1}{(a - x)}$$
Reducimos términos semejantes finalmente:
$$= \frac{1}{1} \times \frac{1}{a} \times \frac{1}{1} \times \frac{1}{1} = \frac{1}{a}$$
  \item $$\frac{(a^2 - 3a)^2}{9 - a^2} \times \frac{27 - a^3}{(a + 3)^2 - 3a} \div \frac{a^4 - 9a^2}{(a^2 + 3a)^2}$$
Iniciamos por cambiar la división en multiplicación intercambiando el numerador y denominador de este factor:
$$= \frac{(a^2 - 3a)^2}{9 - a^2} \times \frac{27 - a^3}{(a + 3)^2 - 3a} \times \frac{(a^2 + 3a)^2}{a^4 - 9a^2}$$
Continuamos expandiendo las restas de cuadrados y la resta de cubos:
$$= \frac{(a^2 - 3a)^2}{(3 + a)(3 - a)} \times \frac{(3 - a)(9 +3a + a^2)}{(a + 3)^2 - 3a} \times \frac{(a^2 + 3a)^2}{(a^2 + 3a)(a^2 - 3a)}$$
Reducimos términos semejantes y expandimos el binomio cuadrado perfecto reduciéndolo con el término independiente:
$$= \frac{a^2 - 3a}{(3 + a)} \times \frac{(9 +3a + a^2)}{(a^2 +6a + 9) - 3a} \times \frac{(a^2 + 3a)}{1}$$
Efectuamos operaciones pendientes y factorizamos términos comunes:
$$= \frac{a(a - 3)}{(3 + a)} \times \frac{1}{1} \times \frac{a(a + 3)}{1}$$
Reducimos términos semejantes:
$$= \frac{a(a - 3)}{1} \times \frac{a}{1} = a^3 - 3a^2$$
\end{enumerate}
\end{document}