\documentclass[12pt]{article}
\usepackage{amsmath}
\usepackage[utf8]{inputenc}
\usepackage[spanish]{babel}
\usepackage{enumitem}

\title{Álgebra. Baldor}
\author{Francisco Treviño zerofrancisco@gmail.com}

\begin{document}
\maketitle

\section*{Ejercicio 89}
Factorar o descomponer en dos factores.\\
Para comprobar que la respuesta es correcta, multiplica los factores del resultado, reacomoda y verifica que tienes la expresión dada en el ejercicio.

\begin{enumerate}[label=\bfseries Ejercicio \arabic*:]
  \item $$a^2 + ab$$
El elemento común a ambos monomios es $a$, y lo utilizamos como factor común lo cual nos da:
$$a(a + b)$$
  \item $$b + b^2$$
El elemento común a ambos monomios es $b$, y lo utilizamos como factor común:
$$b(1+b)$$
  \item $$x^2 + x$$
El elemento común a ambos monomios es $x$, lo utilizamos como factor común, lo cual nos da:
$$x(x+1)$$
  \item $$3a^3 - a^2$$
El elemento común a ambos monomios es $a^2$, al factorizarlo obtenemos:
$$a^2(3a - 1)$$
  \item $$x^3 - 4x^4$$
El elemento común a ambos monomios es $x^3$ el cual factorizamos obteniendo:
$$x^3(1 - 4x)$$
  \item $$5m^2 + 15m^3$$
Factorizamos 5 como máximo común divisor y $m^2$ también como factor común:
$$5m^2(1 + 3m)$$
  \item $$ab - bc$$
El elemento común en ambos monomios es $b$, factorizando tenemos:
$$b(a-c)$$
  \item $$x^2y + x^2z$$
El elemento común a ambos monomios es $x^2$, lo factorizamos para obtener:
$$x^2(y + z)$$
  \item $$2a^2x + 6ax^2$$
El máximo común divisor de ambos monomios es 2 y factorizamos $ax$ porque pertenece a ambos monomios también:
$$2ax(a + 3x)$$
  \item $$8m^2 - 12mn$$
Factorizamos 4 como máximo común divisor de ambos monomios y tomamos $m$ como factor común también:
$$4m(2m - 3n)$$
  \item $$9a^3x^2 - 18ax^2$$
El máximo común divisor de ambos monomios es 9, además $ax^2$ es el monomio presente en ambos, así que lo factorizamos:
$$9ax^2(a^2 -2)$$
  \item $$15c^3d^2 + 60c^2d^3$$
El máximo común divisor es 15 y el monomio presente en ambos monomios es $c^2d^2$, así que factorizando:
$$15c^2d^2(c + 4d)$$
  \item $$35m^2n^3 - 70m^3$$
El máximo común divisor es 35 y el monomio presente en ambos es $m^2$, así que tenemos:
$$35m^2(n^3 - 2m)$$
  \item $$abc + abc^2$$
El monomio que podemos factorizar de ambos monomios originales es $abc$ dado que está presente en ambos:
$$abc(1 + c)$$
  \item $$24a^2xy^2 -36x^2y^4$$
El máximo común divisor es 6 y podemos extraer como monomio común $xy^2$ y obtenemos:
$$6xy^2(4a^2 - 6xy^2)$$
  \item $$a^3 + a^2 + a$$
Podemos factorizar $a$ de todos los monomios obteniendo:
$$a(a^2 + a +1)$$
  \item $$4x^2 - 8x + 2$$
El único factor común entre los tres monomios es 2, factorizando obtenemos:
$$2(2x^2 - 4x + 1)$$
  \item $$15y^3 + 20y^2 - 5y$$
Tenemos como factor común en los tres monomios a $5y$, factorizando obtenemos:
$$5y(3y^2 + 4y - 1)$$
  \item $$a^3 -a^2x + ax^2$$
El único factor común entre los tres monomios es $a$, factorizando obtenemos:
$$a(a^2 - ax + x^2)$$
  \item $$2a^2x +2ax^2 - 3ax$$
El monomio que podemos extraer como elemento común a los originales 3 monomios es $ax$:
$$ax(2a + 2x - 3)$$
  \item $$x^3 +x^5 -x^7$$
Podemos extraer $x^3$ como factor común obteniendo (recuerda que al multiplicar monomios con la misma base, los exponentes se suman):
$$x^3(1 + x^2 - x^4)$$
  \item $$14x^2y^2 -28x^3 + 56x^4$$
Factorizamos 14 como máximo común divisor y tomamos $x^2$ como factor común:
$$14x^2(y^2 - 2x + 4x^2)$$
  \item $$34ax^2 + 51a^2y - 68ay^2$$
No encontramos común divisor y el único factor común entre los monomios originales es $a$:
$$a(34x^2 + 51ay - 68y^2)$$
  \item $$96 - 48mn^2 + 144n^3$$
No tenemos factor común, pero podemos factorizar el máximo común divisor:
$$4(24 - 12mn^2 + 36n^3)$$
  \item $$a^2b^2c^2 - a^2c^2x^2 + a^2c^2y^2$$
Podemos abstraer como factor común $a^2c^2$ dado que se encuentra en los tres monomios originales:
$$a^2c^2(b^2 - x^2 + y^2)$$
  \item $$55m^2n^3x + 110m^2n^3x^2 - 220m^2y^3$$
Tomamos 55 como máximo común divisor, y $m^2$ como factor común entre los monomios originales:
$$55m^2(n^3x + 2 n^3x^2 - 4y^3)$$
  \item $$93a^3x^2y - 62a^2x^3y^2 - 124a^2x$$
Tomamos 31 como máximo común divisor, luego podemos ver que el factor común entre los monomios originales es $a^2x$, factorizando estos dos tenemos:
$$31a^2x(3axy - 2x^2y^2 - 4)$$
  \item $$x - x^2 + x^3 - x^4$$
El único factor común entre los monomios originales es $x$:
$$x(1 - x + x^2 - x^3)$$
  \item $$a^6 - 3a^4 + 8a^3 - 4a^2$$
El único factor común entre los monomios originales es $a^2$:
$$a^2(a^4 - 3a^2 + 8a - 4)$$
  \item $$25x^7 - 10x^5 + 15x^3 - 5x^2$$
El máximo común divisor es 5 y encontramos $x^2$ en todos los monomios, factorizando:
$$5x^2(5x^5 -2x^3 + 3x -1)$$
  \item $$x^{15} - x^{12} + 2x^9 - 3x^6$$
El término común a los monomios es $x^6$, factorizamos recordando que al multiplicar monomios de la misma base los exponentes se suman:
$$x^6(x^9 - x^6 +2x^3 - 3)$$
  \item $$9a^2 - 12ab + 15a^3b^2 - 24 ab^3$$
El máximo común divisor es 3 y el término común a los monomios es $a$, factorizamos:
$$3a(3a - 4b + 5a^2b^2 - 8b^3)$$
  \item $$16x^3y^2 - 8x^2y -24x^4y^2 - 40x^2y^3$$
El máximo común divisor es 8 y el término común a los monomios originales es $x^2y$, factorizamos:
$$8x^2y(2xy - 1 - 3x^2y - 5y^2)$$
  \item $$12m^2n + 24m^3n^2 - 36m^4n^3 + 48m^5n^4$$
El máximo común divisor es 12 y el término común es $m^2n$, factorizamos:
$$12m^2n(1 + 2mn - 3m^2n^2 + 4m^3n^3)$$
  \item $$100a^2b^3c - 150ab^2c^2 + 50ab^3c^3 - 200abc^2$$
El máximo común divisor entre los monomios es 50 y el término común es $abc$, factorizamos:
$$50abc(2ab^2 - 3bc + b^2c^2 - 4c)$$
  \item $$x^5 - x^4 +x^3 - x^2 +x$$
El único factor común entre los monomios originales es $x$, factorizando:
$$x(x^4 - x^3 + x^2 -x + 1)$$
  \item $$a^2 - 2a^3 + 3a^4 - 4a^5 + 6a^6$$
El único factor común entre los monomios es $a^2$, factorizamos:
$$a^2(1 - 2a + 3a^2 - 4a^3 + 6a^4)$$
  \item $$3a^2b + 6ab - 5a^3b^2 + 8a^2bx + 4ab^2m$$
El máximo común divisor es 1 y el término común a todos los monomios es $ab$, factorizamos:
$$ab(3a + 6 -5a^2b + 8ax + 4bm)$$
  \item $$a^{20} - a^{16} + a^{12} - a^8 + a^4 -a^2$$
El término común a todos los monomios es $a^2$, factorizamos:
$$a^2(a^{18} - a^{14} + a^{10} - a^6 + a^2 -1)$$
\end{enumerate}

\section*{Ejercicio 136}
Simplificar:
\begin{enumerate}[label=\bfseries Ejercicio \arabic*:]
  \item $$\frac{3x}{4y} \times \frac{8y}{9x} \div \frac{z^2}{3x^2}$$
Primero cambiamos la división en multiplicación intercambiando el numerador por denominador y viceversa, luego reducimos factores multiplicando respectivamente:
$$= \frac{3x}{4y} \times \frac{8y}{9x} \times \frac{3x^2}{z^2} = \frac{72 x^3y}{36xyz^2}$$
Luego reducimos factores numéricos y factores comunes:
$$= \frac{2x^2}{z^2}$$
  \item $$\frac{5a}{b} \div \left( \frac{2a}{b^2} \times \frac{5x}{4a^2} \right)$$
Primero efectuamos la multiplicación indicada y luego cambiamos la operación de división por una multiplicación intercambiando numerador por denominador y viceversa. Luego efectuamos las multiplicaciones y reducimos:
$$= \frac{5a}{b} \times \frac{4a^2b^2}{10ax} = \frac{2a^2b}{x}$$
  \item $$\frac{a+1}{a-1} \times \frac{3a - 3}{2a + 2} \div \frac{a^2 + a}{a^2 + a - 2}$$
Primero intercambiamos la división por una multiplicación y factorizamos el trinomio y los demás binomios que nos sea posible:
$$= \frac{a+1}{a-1} \times \frac{3a - 3}{2a + 2} \times \frac{a^2 + a - 2}{a^2 + a}$$
$$= \frac{a+1}{a-1} \times \frac{3(a - 1)}{2(a + 1)} \times \frac{(a + 2)(a - 1)}{a (a + 1)}$$
Reduciendo términos semejantes:
$$= \frac{3(a + 2)(a - 1)}{2a(a + 1)} = \frac{3a^2 + 3a - 6}{2a^2 + 2a}$$
  \item $$\frac{64a^2 - 81b^2}{x^2 - 81} \times \frac{(x - 9)^2}{8a - 9b} \div \frac{8a^2 + 9ab}{(x + 9)^2}$$
Primero convertimos la división en multiplicación invirtiendo numerador y denominador y expandemos las operaciones indicadas y restas de binomios al cuadrado condensados:
$$= \frac{64a^2 - 81b^2}{x^2 - 81} \times \frac{(x - 9)^2}{8a - 9b} \times \frac{(x + 9)^2}{8a^2 + 9ab} = \frac{(8a + 9b)(8a - 9b)}{(x + 9)(x - 9)} \times \frac{(x-9)^2}{8a - 9b} \times \frac{(x + 9)^2}{a(8a + 9b)}$$
Reducimos términos semejantes:
$$= \frac{1}{1} \times \frac{(x - 9)}{1} \times \frac{(x + 9)}{a}$$
Finalmente tenemos, reduciendo la resta de cuadrados:
$$= \frac{(x^2 - 81)}{a}$$
  \item $$\frac{x^2 - x - 12}{x^2 - 49} \times \frac{x^2 - x - 56}{x^2 + x - 20} \div \frac{x^2 - 5x - 24}{x + 5}$$
Intercambiamos la división por una multiplicación intercambiando el numerador por denominador y viceversa y expandemos resta de cuadrados y expandemos trinomios:
$$= \frac{x^2 - x - 12}{x^2 - 49} \times \frac{x^2 - x - 56}{x^2 + x - 20} \times \frac{x + 5}{x^2 - 5x - 24}$$
$$= \frac{(x - 4)(x + 3)}{(x + 7)(x - 7)} \times \frac{(x - 8)(x + 7)}{(x + 5)(x - 4)} \times \frac{x+5}{(x - 8)(x + 3)}$$
Reducimos términos semejantes en numerador y denominador:
$$= \frac{1}{x - 7} \times \frac{1}{1} \times \frac{1}{1} = \frac{1}{x - 7} $$
  \item $$\frac{a^2 - 8a + 7}{a^2 - 11a + 30} \times \frac{a^2 - 36}{a^2 - 1} \div \frac{a^2 - a -42}{a^2 - 4a - 5}$$
Primero intercambiamos la división por una multiplicación y factorizamos los trinomios que nos sea posible, también expandemos las sumas de cuadrados por su par de binomios:
$$= \frac{a^2 - 8a + 7}{a^2 - 11a + 30} \times \frac{a^2 - 36}{a^2 - 1} \times \frac{a^2 - 4a - 5}{a^2 - a -42}$$
$$= \frac{(a - 7)(a - 1)}{(a - 6)(a - 5)} \times \frac{(a + 6)(a - 6)}{(a + 1)(a - 1)} \times \frac{(a - 5)(a + 1)}{(a - 7)(a + 6)}$$
Finalmente reducimos términos semejantes en el numerador y denominador:
$$= \frac{1}{1} \times \frac{1}{1} \times \frac{1}{1} = 1$$
  \item $$\frac{x^4 - 27x}{x^2 + 7x - 30} \times \frac{x^2 + 20x + 100}{x^3 + 3x^2 + 9x} \div \frac{x^2 - 100}{x - 3}$$
Expresamos la división como una división intercambiando el numerador y denominador:
$$= \frac{x^4 - 27x}{x^2 + 7x - 30} \times \frac{x^2 + 20x + 100}{x^3 + 3x^2 + 9x} \times \frac{x - 3}{x^2 - 100}$$
expandemos la resta de cubos, si no la recuerdas es un buen momento para repasarla:
$$= \frac{x(x^3 - 27)}{x^2 + 7x - 30} \times \frac{x^2 + 20x + 100}{x^3 + 3x^2 + 9x} \times \frac{x - 3}{x^2 - 100}$$
$$= \frac{x(x - 3)(x^2 + 3x + 9)}{x^2 + 7x - 30} \times \frac{x^2 + 20x + 100}{x^3 + 3x^2 + 9x} \times \frac{x - 3}{x^2 - 100}$$
Factorizamos los trinomios y expandemos la resta de cuadrados:
$$= \frac{x(x - 3)(x^2 + 3x + 9)}{(x + 10)(x - 3)} \times \frac{(x + 10)^2}{x(x^2 + 3x + 9)} \times \frac{x - 3}{(x + 10)(x - 10)}$$
Reducimos términos semejantes en el numerador y denominador:
$$= \frac{1}{1} \times \frac{1}{1} \times \frac{x - 3}{(x - 10)} = \frac{x - 3}{x - 10}$$
  \item $$\frac{(a^2 + 1)}{3a -6} \div \left( \frac{a^3 +a}{6a - 12} \times \frac{4x + 8}{x - 3} \right)$$
Primero realizamos la multiplicación indicada y después cambiamos la división por una multiplicación intercambiando el numerador y denominador:
$$= \frac{(a^2 + 1)}{3a -6} \div \left( \frac{(a^3 +a)(4x + 8)}{(6a - 12)(x - 3)} \right)$$
$$= \frac{(a^2 + 1)}{3a -6} \times \frac{(6a - 12)(x - 3)}{(a^3 +a)(4x + 8)}$$
Hacemos un poco de factorización:
$$= \frac{(a^2 + 1)}{3a -6} \times \frac{2(3a - 6)(x - 3)}{a(a^2 +1)4(x + 2)}$$
Reducimos términos semejantes:
$$= \frac{1}{1} \times \frac{(x - 3)}{2a(x + 2)} = \frac{(x - 3)}{2a(x + 2)}  = \frac{x - 3}{2ax + 4a}$$
  \item $$\frac{8x^2 - 10x -3}{6x^2 +13x + 6} \times \frac{4x^2 - 9}{3x^2 + 2x} \div \frac{8x^2 + 14x + 3}{9x^2 + 12x + 4}$$
Comenzamos por invertir el numerador y denominador para convertir la división en multiplicación:
$$= \frac{8x^2 - 10x -3}{6x^2 +13x + 6} \times \frac{4x^2 - 9}{3x^2 + 2x} \times \frac{9x^2 + 12x + 4}{8x^2 + 14x + 3}$$
Ahora factorizamos los trinomios cuadrados multiplicando y dividiendo por el coeficiente del término cuadrado:
$$8x^2 - 10x -3 = (64x^2 - 10(8)x - 24)(1/8) = (8x - 12)(8x + 2)(1/8)$$
$$ = \frac{8x - 12}{4} \times \frac{8x + 2}{2} = (2x - 3)(4x + 1)$$
El siguiente trinomio tiene solución si aplicamos la fórmula general de la ecuación cuadrática: 
\begin{equation}\label{imp}
  6x^2 +13x + 6
\end{equation}
por lo tanto el trinomio puede factorizarse; aplicamos la fórmula general cuadrática:
$$\frac{-b \pm \sqrt{b^2 - 4ac}}{2a} = \frac{-13 \pm \sqrt{13^2 - 4(6)(6)}}{2(6)} = \frac{-13 \pm \sqrt{169 - 144}}{12}$$
$$= \frac{-13 \pm 5}{12}$$
lo cual nos arroja como resultados $-3/2$ y $-2/3$, esto es:
$$(x + 3/2)(x + 2/3)$$
los cuales multiplicamos por 2 y por 3 respectivamente para quitar coeficientes fraccionarios y tenemos como resultado final:
$$(2x + 3)(3x + 2)$$
el cual podemos verificar multiplicando que nos da el trinomio inicial \ref{imp}.
Tomando otro trinomio del tercer factor para factorizarlo:
$$9x^2 + 12x + 4 = (81x^2 + 12(9)x + 36)(1/9) = (9x + 6)(9x + 6)(1/9)$$
$$\frac{9x + 6}{3} \times \frac{9x + 6}{3} = (3x + 2)^2$$
Tomando el último trinomio del tercer factor tenemos, multiplicando y dividiendo por el coeficiente del término cuadrado:
$$8x^2 + 14x + 3 = (64x^2 + 14(8)x + 24)(1/8)$$
$$= (8x + 12)(8x + 2)(1/8) = \frac{8x + 12}{4} \times \frac{8x + 2}{2}$$
$$= (2x + 3)(4x + 1)$$
Retomando la expresión original sustituyendo los nuevos términos que obtuvimos al factorizar los trinomios tenemos:
$$ = \frac{(2x - 3)(4x + 1)}{(2x + 3)(3x + 2)} \times \frac{4x^2 - 9}{3x^2 + 2x} \times \frac{(3x + 2)^2}{(2x + 3)(4x + 1)}$$
Ahora expandemos la resta de cuadrados y usamos factor común:
$$ = \frac{(2x - 3)(4x + 1)}{(2x + 3)(3x + 2)} \times \frac{(2x + 3)(2x -3)}{x(3x + 2)} \times \frac{(3x + 2)^2}{(2x + 3)(4x + 1)}$$
Finalmente estamos listos para reducir términos semejantes:
$$ = \frac{(2x - 3)}{1} \times \frac{(2x -3)}{x} \times \frac{1}{(2x + 3)} = \frac{4x^2 - 12x + 9}{2x^2 + 3x}$$
  \item $$\frac{(a + b)^2 - c^2}{(a - b)^2 - c^2} \times \frac{(a + c)^2 -b^2}{a^2 + ab -ac} \div \frac{a + b+ c}{a^2}$$
Comenzamos por cambiar la división por una multiplicación intercambiando numerador y denominador:
$$= \frac{(a + b)^2 - c^2}{(a - b)^2 - c^2} \times \frac{(a + c)^2 -b^2}{a^2 + ab -ac} \times \frac{a^2}{a + b+ c}$$
Ahora expandemos las sumas de cuadrados y factorizamos términos comunes:
$$= \frac{[(a+ b) + c][(a+ b) - c]}{[(a - b) + c][(a - b) - c]} \times \frac{[(a + c) + b][(a + c) - b]}{a(a + b - c)} \times \frac{a^2}{a + b+ c}$$
Reducimos términos semejantes:
$$= \frac{1}{[(a - b) - c]} \times \frac{[(a + c) + b]}{1} \times \frac{a}{1}$$
Finalmente tenemos:
$$= \frac{a(a + c + b)}{a - b - c} = \frac{a^2 + ac + ab}{a - b - c}$$
  \item  $$\frac{a^2 - 5a}{b + b^2} \div \left( \frac{a^2 + 6a - 55}{b^2 - 1} \times \frac{ax + 3a}{ab^2 + 11b^2} \right)$$
Empezamos por factorizar el trinomio, expander la resta de cuadrados y factorizar términos comunes en los dos factores del divisor:
$$= \frac{a^2 - 5a}{b + b^2} \div \left( \frac{(a + 11)(a - 5)}{(b+1)(b - 1)} \times \frac{a(x + 3)}{b^2(a + 11)} \right)$$
ahora podemos cambiar numerador y denominador y hacerlo una multiplicación:
$$= \frac{a^2 - 5a}{b + b^2} \times \left( \frac{(b+1)(b - 1)}{(a + 11)(a - 5)} \times \frac{b^2(a + 11)}{a(x + 3)} \right)$$
Factorizamos términos comunes en el primer factor:
$$= \frac{a(a - 5)}{b(1 + b)} \times \left( \frac{(b+1)(b - 1)}{(a + 11)(a - 5)} \times \frac{b^2(a + 11)}{a(x + 3)} \right)$$
Reducimos términos semejantes:
$$= \frac{1}{1} \times \left( \frac{(b - 1)}{1} \times \frac{b}{(x + 3)} \right) = \frac{b^2 - b}{x + 3}$$
  \item $$\frac{m^3 + 6m^2n + 9mn^2}{2m^2n + 7 mn^2 + 3n^3} \times \frac{4m^2 - n^2}{8m^2 - 2mn -n^2} \div \frac{m^3 + 27n^3}{16m^2 + 8mn +n^2}$$
Iniciamos cambiando numerador por denominador para hacer la división una multiplicación
$$= \frac{m^3 + 6m^2n + 9mn^2}{2m^2n + 7 mn^2 + 3n^3} \times \frac{4m^2 - n^2}{8m^2 - 2mn -n^2} \times \frac{16m^2 + 8mn +n^2}{m^3 + 27n^3}$$
Continuamos factorizando términos comunes y expandeendo la resta de cuadrados y también la suma de cubos:
$$= \frac{m(m^2 + 6mn + 9n^2)}{n(2m^2 + 7 mn + 3n^2)} \times \frac{(2m + n)(2m - n)}{8m^2 - 2mn -n^2} \times \frac{16m^2 + 8mn +n^2}{(m + 3n)(m^2 - 3mn + 9n^2)}$$
Factorizando trinomios cuadrados perfectos:
$$= \frac{m(m + 3n)^2}{n(2m^2 + 7mn + 3n^2)} \times \frac{(2m + n)(2m - n)}{8m^2 - 2mn - n^2} \times \frac{(4m + n)^2}{(m + 3n)(m^2 - 3mn + 9n^2)}$$
Ahora factorizamos los trinomios cuadrados que involucran más álgebra:
$$2m^2 + 7mn + 3n^2$$
Multiplicamos y dividimos por el coeficiente del factor del término cuadrático:
$$= (4m^2 + 7(2)mn + 6n^2)(1/2) = (2m + 6n)(2m + n)(1/2)$$
$$= (m + 3n)(2m + n)$$
Ahora hacemos el mismo procedimiento con el siguiente trinomio cuadrado:
$$8m^2 - 2mn - n^2$$
Comenzamos por multiplicar y dividir por el coeficiente del factor del término cuadrático:
$$64m^2 - 2(8)mn - 8n^2 = (8m - 4n)(8m + 2n)(1/8)$$
$$\frac{(8m - 4n)}{4} \frac{(8m + 2n)}{2} = (2m - n)(4m + n)$$
Sustituyendo los trinomios factorizados en la expresión original:
$$= \frac{m(m + 3n)^2}{n(m + 3n)(2m + n)} \times \frac{(2m + n)(2m - n)}{(2m - n)(4m + n)} \times \frac{(4m + n)^2}{(m + 3n)(m^2 - 3mn + 9n^2)}$$
Reducimos términos semejantes:
$$= \frac{m}{n} \times \frac{1}{1} \times \frac{(4m + n)}{(m^2 - 3mn + 9n^2)} = \frac{m(4m + n)}{n(m^2 - 3mn + 9n^2)}$$
Finalmente:
$$= \frac{4m^2 + mn}{m^2n - 3mn^2 + 9n^3}$$
  \item $$\frac{(a^2 - ax)^2}{a^2 + x^2} \times \frac{1}{a^3 + a^2x} \div \left( \frac{a^3 - a^2x}{a^2 + 2ax + x^2} \times \frac{a^2 - x^2}{a^3 + ax^2} \right)$$
Comenzamos por cambiar la división por una multiplicación invirtiendo el numerador y denominadores:
$$= \frac{(a^2 - ax)^2}{a^2 + x^2} \times \frac{1}{a^3 + a^2x} \times \left( \frac{a^2 + 2ax + x^2}{a^3 - a^2x} \times \frac{a^3 + ax^2}{a^2 - x^2} \right)$$
Continuamos por factorizar el trinomio cuadrado perfecto y expander la resta de cuadrados, también factorizamos términos comumes:
$$= \frac{(a^2 - ax)^2}{a^2 + x^2} \times \frac{1}{a^2(a + x)} \times \frac{(a + x)^2}{a(a^2 - ax)} \times \frac{a(a^2 + x^2)}{(a + x)(a - x)}$$
Reducimos términos semejantes:
$$= \frac{(a^2 - ax)}{1} \times \frac{1}{a^2} \times \frac{1}{1} \times \frac{1}{(a - x)}$$
Seguimos factorizando:
$$= \frac{a(a - x)}{1} \times \frac{1}{a^2} \times \frac{1}{1} \times \frac{1}{(a - x)}$$
Reducimos términos semejantes finalmente:
$$= \frac{1}{1} \times \frac{1}{a} \times \frac{1}{1} \times \frac{1}{1} = \frac{1}{a}$$
  \item $$\frac{(a^2 - 3a)^2}{9 - a^2} \times \frac{27 - a^3}{(a + 3)^2 - 3a} \div \frac{a^4 - 9a^2}{(a^2 + 3a)^2}$$
Iniciamos por cambiar la división en multiplicación intercambiando el numerador y denominador de este factor:
$$= \frac{(a^2 - 3a)^2}{9 - a^2} \times \frac{27 - a^3}{(a + 3)^2 - 3a} \times \frac{(a^2 + 3a)^2}{a^4 - 9a^2}$$
Continuamos expandeendo las restas de cuadrados y la resta de cubos:
$$= \frac{(a^2 - 3a)^2}{(3 + a)(3 - a)} \times \frac{(3 - a)(9 +3a + a^2)}{(a + 3)^2 - 3a} \times \frac{(a^2 + 3a)^2}{(a^2 + 3a)(a^2 - 3a)}$$
Reducimos términos semejantes y expandemos el binomio cuadrado perfecto reduciéndolo con el término independiente:
$$= \frac{a^2 - 3a}{(3 + a)} \times \frac{(9 +3a + a^2)}{(a^2 +6a + 9) - 3a} \times \frac{(a^2 + 3a)}{1}$$
Efectuamos operaciones pendientes y factorizamos términos comunes:
$$= \frac{a(a - 3)}{(3 + a)} \times \frac{1}{1} \times \frac{a(a + 3)}{1}$$
Reducimos términos semejantes:
$$= \frac{a(a - 3)}{1} \times \frac{a}{1} = a^3 - 3a^2$$
\end{enumerate}

\section*{Ejercicio 137}
Simplificar:
\begin{enumerate}[label=\bfseries Ejercicio \arabic*:]
  \item $$\frac{a - \frac{a}{b}}{b - \frac{1}{b}}$$
Primero integramos las divisiones en el numerador y denominador en una sola operación:
$$= \frac{\frac{ab - a}{b}}{\frac{b^2 - 1}{b}}$$
Ahora reducimos las dos divisiones en numerador y denominador, multiplicando extremos y medios como se muestra:
$$= \frac{(ab - a)b}{b(b^2 - 1)}$$
Reducimos términos semejantes y expandemos la resta de cuadrados:
$$= \frac{a(b - 1)}{(b + 1)(b - 1)}$$
Finalmente reducimos términos semejantes:
$$= \frac{a}{b + 1}$$
 \item $$\frac{x^2 - \frac{1}{x}}{1 - \frac{1}{x}}$$
Comenzamos por integrar las divisiones en el numerador y denominador en una sola operación:
$$= \frac{\frac{x^3 - 1}{x}}{\frac{x - 1}{x}}$$
Ahora reducimos las dos divisiones en numerador y denominador, multiplicando extremos y medios como se muestra:
$$= \frac{(x^3 - 1)x}{x(x - 1)}$$
Reducimos términos semejantes y expandemos la resta de cubos:
$$= \frac{(x - 1)(x^2 + x +1)}{(x - 1)}$$
Reduciendo términos semejantes tenemos finalmente:
$$= x^2 + x +1$$
  \item $$\frac{\frac{a}{b} - \frac{b}{a}}{1 + \frac{b}{a}}$$
Comenzamos por integrar las operaciones en el numerador y denominador en una sola división cada una:
$$= \frac{\frac{a^2 - b^2}{ab}}{\frac{a + b}{a}}$$
Ahora reducimos las dos divisiones en numerador y denominador, multiplicando extremos y medios como se muestra:
$$= \frac{(a^2 - b^2)a}{ab(a + b)}$$
Reducimos términos semejantes y expandemos la resta de cuadrados:
$$= \frac{(a + b)(a - b)}{b(a + b)}$$
Reduciendo términos semejantes tenemos finalmente:
$$= \frac{a - b}{b}$$
\item $$\frac{\frac{1}{m} + \frac{1}{n}}{\frac{1}{m} - \frac{1}{n}}$$
Comenzamos por integrar las operaciones en el numerador y denominador en una sola división cada una:
$$= \frac{\frac{n + m}{mn}}{\frac{n - m}{mn}}$$
Ahora reducimos las dos divisiones en numerador y denominador, multiplicando extremos y medios como se muestra:
$$= \frac{(n + m)mn}{mn(n-m)}$$
Reduciendo términos semejantes tenemos finalmente:
$$= \frac{m + n}{n-m}$$
  \item $$\frac{x + \frac{x}{2}}{x - \frac{x}{4}}$$
Comenzamos por integrar las operaciones en el numerador y denominador en una sola división cada una:
$$= \frac{\frac{2x + x}{2}}{\frac{4x - x}{4}}$$
Ahora reducimos las dos divisiones en numerador y denominador, multiplicando extremos y medios como se muestra:
$$= \frac{(2x + x)4}{2(4x - x)}$$
Reduciendo términos tenemos finalmente:
$$= \frac{2(2x + x)}{4x - x} = \frac{2(3x)}{3x} = 2$$
  \item $$\frac{\frac{x}{y} - \frac{y}{x}}{1 + \frac{y}{x}}$$
Comenzamos por integrar las operaciones en el numerador y denominador en una sola división cada una:
$$= \frac{\frac{x^2 - y^2}{xy}}{\frac{x + y}{x}}$$
Ahora reducimos las dos divisiones en numerador y denominador, multiplicando extremos y medios como se muestra:
$$= \frac{(x^2 - y^2)x}{xy(x + y)}$$
Reducimos términos semejantes y expandemos la resta de cuadrados:
$$= \frac{(x + y)(x - y)}{y(x + y)}$$
Reduciendo términos semejantes tenemos finalmente:
$$= \frac{x - y}{y}$$
  \item $$\frac{x + 4 +\frac{3}{x}}{x - 4 - \frac{5}{x}}$$
Comenzamos por integrar las operaciones en el numerador y denominador en una sola división cada una:
$$= \frac{\frac{x^2 + 4x +3}{x}}{\frac{x^2 - 4x - 5}{x}}$$
Ahora reducimos las dos divisiones en numerador y denominador, multiplicando extremos y medios como se muestra:
$$= \frac{(x^2 + 4x +3)x}{x(x^2 - 4x - 5)}$$
Reducimos términos semejantes y factorizamos los trinomios cuadrados:
$$= \frac{(x + 3)(x + 1)}{(x - 5)(x +1)}$$
Reduciendo términos semejantes tenemos finalmente:
$$= \frac{x + 3}{x - 5}$$
  \item $$\frac{a - 4 + \frac{4}{a}}{1 - \frac{2}{a}}$$
Comenzamos por integrar las operaciones en el numerador y denominador en una sola división cada una:
$$= \frac{\frac{a^2 - 4a + 4}{a}}{\frac{a - 2}{a}}$$
Ahora reducimos las dos divisiones en numerador y denominador, multiplicando extremos y medios como se muestra:
$$= \frac{(a^2 - 4a + 4)a}{a(a - 2)}$$
Reducimos términos semejantes y factorizamos el trinomio cuadrado perfecto:
$$= \frac{(a - 2)^2}{a - 2} = a - 2$$
  \item $$\frac{\frac{2a^2 - b ^2}{a} - b}{\frac{4a^2 + b^2}{4ab} + 1}$$
Comenzamos por integrar las operaciones en el numerador y denominador en una sola división cada una:
$$= \frac{\frac{2a^2 - b^2 - ab}{a}}{\frac{4a^2 + b^2 + 4ab}{4ab}}$$
Ahora reducimos las dos divisiones en numerador y denominador, multiplicando extremos y medios como se muestra:
$$= \frac{(2a^2 - b^2 - ab)4ab}{a(4a^2 + b^2 + 4ab)}$$
Factorizamos el trinomio cuadrado del numerador como sigue. Primero multiplicamos y dividimos por el coeficiente del término cuadrático:
$$2a^2 - ab - b^2 = (4a^2 -(2)ab - 2b^2)(1/2)$$
Luego factorizamos y dividimos:
$$(2a - 2b)(2a + b)(1/2) = (a - b)(2a + b)$$
El trinomio del denominador es cuadrado perfecto, se factoriza fácilmente. Sustituyendo este par de resultados tenemos:
$$= \frac{(a - b)(2a + b)4ab}{a(2a + b)^2}$$
Reducimos términos semejantes:
$$= \frac{4b(a - b)}{2a + b} = \frac{4ab - 4b^2}{2a + b}$$
  \item $$\frac{2 + \frac{3a}{5b}}{a + \frac{10b}{3}}$$
Comenzamos por integrar las operaciones en el numerador y denominador en una sola división cada una:
$$= \frac{\frac{10b + 3a}{5b}}{\frac{3a + 10b}{3}}$$
Ahora reducimos las dos divisiones en numerador y denominador, multiplicando extremos y medios como se muestra y reducimos términos semejantes:
$$= \frac{(10b + 3a)3}{5b(3a + 10b)} = \frac{3}{5b}$$
  \item $$\frac{a - x + \frac{x^2}{a + x}}{a^2 - \frac{a^2}{a + x}}$$
Comenzamos por integrar las operaciones en el numerador y denominador en una sola división cada una:
$$= \frac{\frac{a(a + x) -x(a + x) + x^2}{a + x}}{\frac{a^2(a + x) - a^2}{a + x}}$$
Podemos reducir términos semejantes:
$$= \frac{a(a + x) -x(a + x) + x^2}{a^2(a + x) - a^2}$$
Realizamos las operaciones indicadas en el numerador y factorizamos términos comunes en el denominador:
$$= \frac{(a^2 + ax) - (ax + x^2) + x^2}{a^2[(a + x) - 1]} = \frac{a^2}{a^2(a + x - 1)}$$
Reduciendo finalmente:
$$= \frac{1}{a + x - 1}$$
  \item $$\frac{a + 5 - \frac{14}{a}}{1 + \frac{8}{a} + \frac{7}{a^2}}$$
Comenzamos por integrar las operaciones en el numerador y denominador en una sola división cada una utilizando el mínimo común múltiplo:
$$= \frac{\frac{a^2 + 5a - 14}{a}}{\frac{a^2 + 8a + 7}{a^2}}$$
Ahora reducimos las dos divisiones en numerador y denominador, multiplicando extremos y medios como se muestra y reducimos términos semejantes:
$$= \frac{(a^2 + 5a - 14)a}{a^2 + 8a + 7}$$
Factorizamos los trinomios cuadrados del numerador y denominador:
$$= \frac{a(a + 7)(a - 2)}{(a + 7)(a + 1)}$$
Reduciendo finalmente:
$$= \frac{a(a - 2)}{a + 1} = \frac{a^2 - 2a}{a + 1}$$
  \item $$\frac{\frac{1}{a} - \frac{9}{a^2} + \frac{20}{a^3}}{\frac{16}{a} - a}$$
Comenzamos por integrar las operaciones en el numerador y denominador en una sola división cada una utilizando el mínimo común múltiplo:
$$= \frac{\frac{a^2 - 9a + 20}{a^3}}{\frac{16 - a^2}{a}}$$
Ahora reducimos las dos divisiones en numerador y denominador, multiplicando extremos y medios como se muestra y reducimos términos semejantes:
$$= \frac{a^2 - 9a + 20}{a^2(16 - a^2)}$$
Factorizamos el trinomio cuadrado y expandemos la resta de cuadrados:
$$= \frac{(a - 5)(a - 4)}{a^2(4 + a)(4 - a)}$$
Podemos reducir cambiando el signo a la fracción en numerador y denominador:
$$= \frac{(5 - a)}{a^2(4 + a)} = \frac{5 - a}{a^3 + 4a^2}$$
  \item $$\frac{\frac{20x^2 + 7x - 6}{x}}{\frac{4}{x^2} - 25}$$
Comenzamos por integrar las operaciones en el denominador en una sola división utilizando el mínimo común múltiplo:
$$= \frac{\frac{20x^2 + 7x - 6}{x}}{\frac{4 - 25x^2}{x^2}}$$
Ahora reducimos las dos divisiones en numerador y denominador, multiplicando extremos y medios como se muestra y reducimos términos semejantes:
$$= \frac{(20x^2 + 7x - 6)x}{4 - 25x^2}$$
Ahora factorizamos el trinomio cuadrado de la siguiente manera. Primero multiplicamos todo el trinomio por el coeficiente del término cuadrático (indicando solamente la multiplicación en el segundo término), luego buscamos dos números que restados nos den el segundo término (el original, sin multiplicar) y multiplicados nos den el tercer término:
$$20x^2 + 7x - 6 = (400x^2 + 7(20)x -120)(1/20)$$
Factorizamos este nuevo trinomio:
$$= \frac{(20x + 15)}{5}\frac{(20x - 8)}{4} = (4x + 3)(5x -2)$$
Expresamos de nuevo la expresión original con esta nueva factorización del trinomio y expandemos la resta de cuadrados:
$$= \frac{(20x^2 + 7x - 6)x}{4 - 25x^2} = \frac{(4x + 3)(5x -2)x}{(2 + 5x)(2 - 5x)}$$
Cambiamos de signo la fracción tanto en el numerador como en el denominador para reducir términos:
$$= \frac{-x (4x + 3)}{2 + 5x} = - \frac{4x^2 + 3x}{5x + 2}$$
  \item $$\frac{1 + \frac{1}{x - 1}}{1 + \frac{1}{x^2 - 1}}$$
Comenzamos por integrar las operaciones en el numerador y denominador en una sola división cada una utilizando el mínimo común múltiplo:
$$= \frac{\frac{x - 1 + 1}{x - 1}}{\frac{x^2 - 1 + 1}{x^2 - 1}}$$
Reduciendo operaciones y reducimos las dos divisiones en numerador y denominador, multiplicando extremos y medios como se muestra:
$$= \frac{\frac{x}{x - 1}}{\frac{x^2}{x^2 - 1}} = \frac{x(x^2 - 1)}{(x - 1)x^2}$$
expandeendo la resta de cuadrados y reduciendo términos semejantes:
$$= \frac{(x + 1)(x - 1)}{x(x - 1)}$$
Reduciendo finalmente:
$$= \frac{x + 1}{x}$$
  \item $$\frac{a - \frac{ab}{a + b}}{a + \frac{ab}{a - b}}$$
Comenzamos por integrar las operaciones en el numerador y denominador en una sola división utilizando el mínimo común múltiplo:
$$= \frac{\frac{a(a+b) - ab}{a + b}}{\frac{a(a - b) + ab}{a - b}}$$
Efectuamos operaciones y reduciendo operaciones reducimos las dos divisiones en numerador y denominador, multiplicando extremos y medios como se muestra:
$$= \frac{(a^2 +ab - ab)(a - b)}{(a + b)(a^2 - ab + ab)} = \frac{(a^2)(a - b)}{(a + b)(a^2)}$$
Reduciendo finalmente:
$$= \frac{a - b}{a + b}$$
  \item $$\frac{x - 1 - \frac{5}{x + 3}}{x + 5 - \frac{35}{x + 3}}$$
Comenzamos por integrar las operaciones en el numerador y denominador en una sola división cada una utilizando el mínimo común múltiplo:
$$= \frac{\frac{(x - 1)(x + 3) - 5}{x + 3}}{\frac{(x + 5)(x + 3) - 35}{x + 3}}$$
Reducimos términos semejantes:
$$= \frac{(x - 1)(x + 3) - 5}{(x + 5)(x + 3) - 35}$$
expandemos las operaciones:
$$= \frac{x^2 + 3x -x - 3 - 5}{x^2 + 3x + 5x + 15 -35} = \frac{x^2 + 2x - 8}{x^2 + 8x - 20}$$
Ahora factorizamos los trinomios cuadrados y reducimos términos semejantes:
$$= \frac{(x + 4)(x - 2)}{(x + 10)(x - 2)} = \frac{x + 4}{x + 10}$$
  \item $$\frac{a + 2 - \frac{7a + 9}{a + 3}}{a - 4 + \frac{5a - 11}{a + 1}}$$
Comenzamos por integrar las operaciones en el numerador y denominador en una sola división utilizando el mínimo común múltiplo:
$$= \frac{\frac{(a + 2)(a + 3) - (7a + 9)}{a + 3}}{\frac{(a - 4)(a + 1) + 5a - 11}{a + 1}}$$
Efectuamos operaciones e indicamos nuevas operaciones, también reducimos las dos divisiones en numerador y denominador, multiplicando extremos y medios como se muestra:
$$= \frac{(a^2 + 3a + 2a + 6 -7a -9)(a + 1)}{(a^2 + a - 4a - 4 + 5a - 11)(a + 3)} = \frac{(a^2 - 2a - 3)(a + 1)}{(a^2 + 2a - 15)(a + 3)}$$
Factorizamos los trinomios y reducimos términos semejantes:
$$= \frac{(a - 3)(a + 1)(a + 1)}{(a + 5)(a - 3)(a + 3)}= \frac{(a + 1)(a + 1)}{(a + 5)(a + 3)}$$
Finalmente expandemos las operaciones:
$$= \frac{a^2 + 2a + 1}{a^2 + 8a +15}$$
\end{enumerate}

\section*{Ejercicio 138}
Simplificar:
\begin{enumerate}[label=\bfseries Ejercicio \arabic*:]
  \item $$\frac{1 + \frac{x + 1}{x - 1}}{\frac{1}{x - 1} - \frac{1}{x + 1}}$$
Comenzamos por integrar las operaciones en el numerador y denominador en una sola división cada una utilizando el mínimo común múltiplo:
$$= \frac{\frac{x - 1 + x + 1}{x - 1}}{\frac{x + 1 - (x - 1)}{(x - 1)(x + 1)}}$$
Efectuamos operaciones y reduciendo operaciones reducimos las dos divisiones en numerador y denominador, multiplicando extremos y medios como se muestra. Finalmente reducimos términos semejantes:
$$= \frac{2x(x - 1)(x + 1)}{2(x-1)} = x(x + 1) = x ^2 + x$$
  \item $$\frac{\frac{1}{x - 1} + \frac{2}{x + 1}}{\frac{x - 2}{x} + \frac{2x + 6 }{x + 1}}$$
Comenzamos por integrar las operaciones en el numerador y denominador en una sola división cada una utilizando el mínimo común múltiplo:
$$= \frac{\frac{x + 1 + 2(x -1)}{(x - 1)(x + 1)}}{\frac{(x - 2)(x + 1)+ (2x + 6)(x)}{x(x + 1)}}$$
Efectuamos operaciones y reduciendo operaciones reducimos las dos divisiones en numerador y denominador, multiplicando extremos y medios como se muestra, también reducimos términos semejantes:
$$= \frac{(3x - 1)x}{(x - 1)(x^2 + x - 2x -2 + 2x^2 + 6x)} = \frac{x(3x - 1)}{(x - 1)(3x^2 + 5x - 2)}$$
Factorizamos el trinomio cuadrado del denominador multiplicando (y dividiendo) primero sus términos por el coeficiente del término cuadrático y luego factorizamos como se muestra:
$$3x^2 + 5x - 2 = (9x^2 + 5(3)x - 6)(1/3)$$
$$= (3x + 6)(3x - 1)(1/3) = (x + 2)(3x - 1)$$
Sustituimos en la expresión original este resultado y reducimos términos semejantes:
$$= \frac{x(3x - 1)}{(x - 1)(x + 2)(3x - 1)} = \frac{x}{(x - 1)(x + 2)}$$
$$= \frac{x}{x^2 + x - 2}$$
  \item $$\frac{\frac{a}{a - b} - \frac{b}{a + b}}{\frac{a + b}{a - b} + \frac{a}{b}}$$
Comenzamos por integrar las operaciones en el numerador y denominador en una sola división cada una utilizando el mínimo común múltiplo:
$$= \frac{\frac{a(a + b) - b(a - b)}{(a - b)(a + b)}}{\frac{(a + b)b + a(a - b)}{(a - b)b}}$$
Realizamos operaciones y reducimos las dos divisiones en numerador y denominador, multiplicando extremos y medios como se muestra:
$$= \frac{(a^2 + b^2)b}{(a^2 + b^2)(a + b)} = \frac{b}{a + b}$$
  \item $$\frac{\frac{x + 3}{x + 4} - \frac{x + 1}{x + 2}}{\frac{x - 1}{x + 2} - \frac{x - 3}{x + 4}}$$
Comenzamos por integrar las operaciones en el numerador y denominador en una sola división cada una utilizando el mínimo común múltiplo:
$$= \frac{\frac{(x + 3)(x + 2) - (x + 1)(x + 4)}{(x + 4)(x + 2)}}{\frac{(x - 1)(x + 4) - (x - 3)(x + 2)}{(x + 2)(x + 4)}}$$
Observamos que podemos reducir los denominadores y realizamos operaciones:
$$= \frac{(x + 3)(x + 2) - (x + 1)(x + 4)}{(x - 1)(x + 4) - (x - 3)(x + 2)} = \frac{x^2 + 2x + 3x +6 -  x^2 - 4x - x - 4}{x^2 + 4x - x - 4 - x^2 - 2x + 3x + 6}$$
Realizando operaciones:
$$= \frac{2}{4x + 2} = \frac{1}{2x + 1}$$
  \item $$\frac{\frac{m^2}{n} - \frac{m^2 - n^2}{m + n}}{\frac{m - n}{n} + \frac{n}{m}}$$
Comenzamos por integrar las operaciones en el numerador y denominador en una sola división cada una utilizando el mínimo común múltiplo:
$$= \frac{\frac{m^2(m + n) - (m^2 - n^2)n}{n(m + n)}}{\frac{(m - n)m + n(n)}{nm}}$$
Realizamos operaciones y reducimos las dos divisiones en numerador y denominador, multiplicando extremos y medios como se muestra, reduciendo términos donde es posible:
$$= \frac{(m^3 + m^2n - m^2n + n^3)m}{(m + n)(m^2 - mn + n^2)} = \frac{m(m^3 + n^3)}{(m + n)(m^2 - mn + n^2)}$$
expandeendo la suma de cubos y finalmente reduciendo:
$$= \frac{m(m + n)(m^2 - mn + n^2)}{(m + n)(m^2 - mn + n^2)} = m$$
  \item $$\frac{\frac{a^2}{b^3} + \frac{1}{a}}{\frac{a}{b} - \frac{b - a}{a - b}}$$
Comenzamos por integrar las operaciones en el numerador y denominador en una sola división cada una utilizando el mínimo común múltiplo:
$$= \frac{\frac{a^2(a) + b^3}{ab^3}}{\frac{a}{b} - \frac{b - a}{a - b}}$$
En el segundo factor del denominador cambiamos el signo al numerador y al signo de la fracción para reducir términos:
$$= \frac{\frac{a^2(a) + b^3}{ab^3}}{\frac{a}{b} + \frac{a - b}{a - b}} = \frac{\frac{a^2(a) + b^3}{ab^3}}{\frac{a}{b} + 1}$$
Luego convertimos el denominador en una sola fracción tal como hicimos con el numerador y convertimos en una sola división multiplicando extremos por extremos y medios por medios, reduciendo términos comunes:
$$= \frac{\frac{a^2(a) + b^3}{ab^3}}{\frac{a + b}{b}} = \frac{a^3 + b^3}{ab^2(a + b)}$$
expandemos la suma de cubos y reduciendo:
$$= \frac{(a + b)(a^2 - ab + b^2)}{ab^2(a + b)} = \frac{a^2 - ab + b^2}{ab^2}$$
  \item $$\frac{1 + \frac{2x}{1 + x^2}}{2x + \frac{2x^5 + 2}{1 - x^4}}$$
Comenzamos por integrar las operaciones en el numerador y denominador en una sola división cada una utilizando el mínimo común múltiplo:
$$= \frac{\frac{1 + x^2 + 2x}{1 + x^2}}{\frac{2x(1 - x^4)+ 2x^5 + 2}{1 - x^4}}$$
expandemos resta de cuadrados y reducimos efectuando operaciones:
$$= \frac{\frac{1 + x^2 + 2x}{1 + x^2}}{\frac{2x(1 - x^4)+ 2x^5 + 2}{(1 + x^2)(1 - x^2)}} = \frac{\frac{1 + x^2 + 2x}{1}}{\frac{2x - 2x^5+ 2x^5 + 2}{1 - x^2}} = \frac{1 + x^2 + 2x}{\frac{2x + 2}{1 - x^2}}$$
expandemos resta de cuadrados y factorizamos:
$$= \frac{x^2 + 2x +1}{\frac{2(x + 1)}{(1 + x)(1 - x)}} = \frac{x^2 + 2x +1}{\frac{2}{(1 - x)}}$$
Expresamos en una sola fracción:
$$= (x^2 + 2x +1)(1-x)(1/2)$$
Realizamos las operaciones:
$$= (x^2 + 2x +1 - x^3 - 2x^2 - x)(1/2) = (1 + x - x^2 - x^3)(1/2)$$
  \item $$\frac{\frac{x + y}{x - y} - \frac{x - y}{x + y}}{\frac{x + y}{x} - \frac{x + 2y}{x + y}}$$
Comenzamos por integrar las operaciones en el numerador y denominador en una sola división cada una utilizando el mínimo común múltiplo:
$$= \frac{\frac{(x + y)(x + y) - (x - y)(x - y)}{(x - y)(x + y)}}{\frac{(x + y)(x + y) - (x + 2y)x}{x(x + y)}}$$
Reducimos las divisiones en numerador y denominador multiplicando extremos y medios, reduciendo términos semejantes:
$$= \frac{[(x + y)(x + y) - (x - y)(x - y)]x}{(x - y)[(x + y)(x + y) - (x + 2y)x]}$$
Realizamos las operaciones:
$$= \frac{(x^2 + 2xy + y^2 - x^2 + 2xy - y^2)x}{(x - y)(x^2 + 2xy + y^2 - x^2 - 2xy)} = \frac{(4xy)x}{(x - y)(y^2)}$$
Finalmente:
$$= \frac{4x^2}{xy - y^2}$$
  \item $$\frac{\frac{a + x}{a - x} - \frac{b + x}{b - x}}{\frac{2}{a - x} - \frac{2}{b -x}}$$
Comenzamos por integrar las operaciones en el numerador y denominador en una sola división cada una utilizando el mínimo común múltiplo:
$$= \frac{\frac{(a + x)(b - x) - (b + x)(a - x)}{(a - x)(b - x)}}{\frac{2(b - x) - 2(a - x)}{(a - x)(b - x)}}$$
Reducimos términos semejantes:
$$= \frac{(a + x)(b - x) - (b + x)(a - x)}{2(b - x) - 2(a - x)}$$
Realizamos operaciones y reducimos:
$$= \frac{ab - ax + bx - x^2 - ab + bx - ax + x^2}{2b - 2x - 2a + 2x} = \frac{-2ax + 2bx }{2(b - a)}$$
Factorizamos y reducimos términos semejantes:
$$= \frac{2x(b - a)}{2(b - a)} = x$$
  \item $$\frac{\frac{a}{a + x} - \frac{a}{2a + 2x}}{\frac{a}{a - x} + \frac{a}{a + x}}$$
Comenzamos por integrar las operaciones en el numerador y denominador en una sola división cada una utilizando el mínimo común múltiplo:
$$= \frac{\frac{a(2a + 2x) - a(a + x)}{(a + x)(2a + 2x)}}{\frac{a(a + x) + a(a - x)}{(a - x)(a + x)}}$$
Reducimos las divisiones en numerador y denominador multiplicando extremos y medios, reduciendo términos semejantes y realizando operaciones:
$$= \frac{(2a^2 + 2ax - a^2 - ax)(a - x)}{2(a + x)(a^2 + ax + a^2 - ax)} = \frac{(a^2 + ax)(a - x)}{2(a + x)(2a^2)}$$
Simplificando:
$$= \frac{a(a + x)(a - x)}{4a^2(a + x)} = \frac{a - x}{4a}$$
  \item $$\frac{\frac{a + 2b}{a - b}+ \frac{b}{a}}{\frac{a + b}{a} + \frac{3b}{a - b}}$$
Comenzamos por integrar las operaciones en el numerador y denominador en una sola división cada una utilizando el mínimo común múltiplo:
$$= \frac{\frac{(a + 2b)a + b(a - b)}{(a - b)a}}{\frac{(a + b)(a - b) + 3ba}{a(a - b)}}$$
Eliminamos términos semejantes y realizamos operaciones:
$$= \frac{a^2 + 2ab + ab - b^2}{a^2 -ab + ab - b^2 + 3ab} = \frac{a^2 + 3ab - b^2}{a^2 + 3ab - b^2} = 1$$
  \item $$\frac{1 - \frac{7}{x} + \frac{12}{x^2}}{x - \frac{16}{x}}$$
Comenzamos por integrar las operaciones en el numerador y denominador en una sola división cada una utilizando el mínimo común múltiplo:
$$= \frac{\frac{x^2 - 7x + 12}{x^2}}{\frac{x^2 - 16}{x}}$$
Eliminamos términos semejantes, factorizamos el trinomio cuadrado y expandemos la diferencia de cuadrados:
$$= \frac{(x - 4)(x - 3)}{x(x + 4)(x - 4)} = \frac{x - 3}{x(x + 4)} = \frac{x - 3}{x^2 + 4x}$$
  \item $$\frac{\frac{a^2}{b} - \frac{b^2}{a}}{\frac{1}{b} + \frac{1}{a} + \frac{b}{a^2}}$$
Comenzamos por integrar las operaciones en el numerador y denominador en una sola división cada una utilizando el mínimo común múltiplo:
$$= \frac{\frac{a^2a - b^2b}{ab}}{\frac{a^2 + ab + b^2}{a^2b}}$$
Reducimos términos semejantes y efectuamos operaciones, convertimos en una sola división multiplicando extremos y medios:
$$= \frac{(a^3 - b^3)a}{a^2 + ab + b^2}$$
expandemos la diferencia de cubos y reducimos términos semejantes:
$$= \frac{a(a - b)(a^2 + ab + b^2)}{a^2 + ab + b^2} = a(a - b) = a^2 - ab$$
  \item $$\frac{x - 2y - \frac{4y^2}{x + y}}{x - 3y - \frac{5y^2}{x + y}}$$
Comenzamos por integrar las operaciones en el numerador y denominador en una sola división cada una utilizando el mínimo común múltiplo:
$$= \frac{\frac{(x - 2y)(x + y) - 4y^2}{x + y}}{\frac{(x - 3y)(x + y) - 5y^2}{x + y}}$$
Reduciendo términos semejantes y realizando operaciones:
$$= \frac{x^2 + xy - 2xy - 2y^2 -4y^2}{x^2 + xy - 3xy - 3y^2 - 5y^2} = \frac{x^2 - xy - 6y^2}{x^2 - 2xy -8y^2}$$
Factorizando los trinomios cuadrados:
$$= \frac{(x - 3y)(x + 2y)}{(x - 4y)(x + 2y)} = \frac{x - 3y}{x - 4y}$$
  \item $$\frac{\frac{2}{1 - a} + \frac{2}{1 + a}}{\frac{2}{1 + a} - \frac{2}{1 - a}}$$
Comenzamos por integrar las operaciones en el numerador y denominador en una sola división cada una utilizando el mínimo común múltiplo:
$$= \frac{\frac{2(1 + a) + 2(1 - a)}{(1 - a)(1 + a)}}{\frac{2(1 - a) - 2(1 + a)}{(1 + a)(1 - a)}}$$
Reducimos términos semejantes y realizamos operaciones:
$$= \frac{2 + 2a + 2 - 2a}{2 - 2a - 2 - 2a} = \frac{4}{-4a} = - \frac{1}{a}$$
  \item $$\frac{\frac{1}{x + y + z} - \frac{1}{x - y + z}}{\frac{1}{x - y + z} - \frac{1}{x + y + z}}$$
Cambiamos el signo de la ecuación completa y el signo del numerador para no afectar el signo en general:
$$= - \frac{- \frac{1}{x + y + z} + \frac{1}{x - y + z}}{\frac{1}{x - y + z} - \frac{1}{x + y + z}}$$
El numerador y el denominador son completamente iguales:
$$= - 1$$
  \item $$\frac{1 + \frac{2b + c}{a - b - c}}{1 - \frac{c - 2b}{a - b + c}}$$
Comenzamos por integrar las operaciones en el numerador y denominador en una sola división cada una utilizando el mínimo común múltiplo:
$$= \frac{\frac{a - b- c + 2b + c}{a - b - c}}{\frac{a - b + c - c +2b}{a - b + c}}$$
Efectuamos operaciones, convertimos en una sola división multiplicando extremos y medios:
$$= \frac{(a + b)(a - b + c)}{(a + b)(a - b - c)} = \frac{a - b + c}{a - b - c}$$
  \item $$\frac{\frac{a}{1 - a} + \frac{1 - a}{a}}{\frac{1 - a}{a} - \frac{a}{1 - a}}$$
Comenzamos por integrar las operaciones en el numerador y denominador en una sola división cada una utilizando el mínimo común múltiplo:
$$= \frac{\frac{a^2 + (1 - a)(1 - a)}{a(1 - a)}}{\frac{(1 - a)(1 - a) - a^2}{a(1 - a)}}$$
Reducimos términos semejantes y efectuamos operaciones:
$$= \frac{a^2 + (1 - a)(1 - a)}{(1 - a)(1 - a) - a^2} = \frac{a^2 + 1 - 2a + a^2}{1 - 2a + a^2 - a^2}$$
$$= \frac{2a^2 - 2a + 1}{1 - 2a}$$
  \item $$\frac{\left( x + 1 - \frac{6x + 12}{x + 2} \right) / (x - 5)}{\left( x - 4 + \frac{11x - 22}{x - 2} \right) / (x + 7)}$$
Primero atacamos las expresiones que contienen monomios y una fracción tanto en el numerador como en el denominador:
$$= \frac{\left( x + 1 - \frac{6(x + 2)}{x + 2} \right) / (x - 5)}{\left( x - 4 + \frac{11(x - 2)}{x - 2} \right) / (x + 7)}$$
Reducimos:
$$= \frac{\left( x + 1 - 6 \right) / (x - 5)}{\left( x - 4 + 11 \right) / (x + 7)} = \frac{(x - 5) / (x - 5)}{(x + 7) / (x + 7)} = 1$$
  \item $$\frac{1}{1 + \frac{1}{x}}$$
Comenzamos por integrar las operaciones en el denominador en una sola división utilizando el mínimo común múltiplo y simplificamos:
$$= \frac{1}{\frac{x + 1}{x}} = \frac{x}{x + 1}$$
  \item $$\frac{1}{1 + \frac{1}{1 - \frac{1}{x}}}$$
Comenzamos a expresar como una sola fracción a partir del denominador más embebido:
$$= \frac{1}{1 + \frac{1}{\frac{x - 1}{x}}} = \frac{1}{1 + \frac{x}{x - 1}}$$
Continuamos a expresar como una sola fracción a partir del denominador más embebido:
$$= \frac{1}{\frac{x - 1 + x}{x - 1}} = \frac{1}{\frac{2x - 1}{x - 1}} = \frac{x - 1}{2x - 1}$$
  \item $$1 - \frac{1}{2 + \frac{1}{\frac{x}{3} - 1}}$$
Comenzamos a expresar como una sola fracción a partir del denominador más embebido:
$$= 1 - \frac{1}{2 + \frac{1}{\frac{x- 3}{3}}} = 1 - \frac{1}{2 + \frac{3}{x- 3}}$$
Continuamos a expresar como una sola fracción a partir del denominador más embebido:
$$= 1 - \frac{1}{\frac{2(x - 3) + 3}{x- 3}} = 1 - \frac{x- 3}{2x - 3}$$
Expresamos como una sola fracción:
$$= \frac{2x - 3 - x + 3}{2x - 3} = \frac{x}{2x - 3}$$
  \item $$\frac{2}{1 + \frac{2}{1 + \frac{2}{x}}}$$
Comenzamos a expresar como una sola fracción a partir del denominador más embebido:
$$= \frac{2}{1 + \frac{2}{\frac{x + 2}{x}}} = \frac{2}{1 + \frac{2x}{x + 2}}$$
Continuamos a expresar como una sola fracción a partir del denominador más embebido:
$$= \frac{2}{\frac{x + 2 + 2x}{x + 2}} = \frac{2}{\frac{3x + 2}{x + 2}} = \frac{2x + 4}{3x + 2}$$
  \item $$\frac{1}{x - \frac{x}{x - \frac{x^2}{x + 1}}}$$
Comenzamos a expresar como una sola fracción a partir del denominador más embebido:
$$= \frac{1}{x - \frac{x}{\frac{x(x + 1) - x^2}{x + 1}}}$$
Realizamos las operaciones:
$$= \frac{1}{x - \frac{x}{\frac{x^2 + x - x^2}{x + 1}}} = \frac{1}{x - \frac{x}{\frac{x}{x + 1}}}$$
Trabajamos en expresar como una sola fracción a partir del denominador más embebido:
$$= \frac{1}{x - \frac{x(x + 1)}{x}} = \frac{1}{x - x - 1} = - 1$$
  \item $$\frac{1}{a + 2 - \frac{a + 1}{a - \frac{1}{a}}}$$
Comenzamos a expresar como una sola fracción a partir del denominador más embebido:
$$= \frac{1}{a + 2 - \frac{a + 1}{\frac{a^2 - 1}{a}}}$$
Nos deshacemos de una división multiplicando extremos y medios y hacemos operaciones y expandemos la suma de  cuadrados, luego reducimos términos semejantes:
$$= \frac{1}{a + 2 - \frac{(a + 1)a}{(a + 1)(a - 1)}} = \frac{1}{a + 2 - \frac{a}{a - 1}}$$
Trabajamos en expresar como una sola fracción a partir del denominador:
$$= \frac{1}{\frac{(a + 2)(a - 1) - a}{a - 1}}$$
Escribimos como una sola fracción y realizamos operaciones:
$$= \frac{a - 1}{a^2 - a + 2a -2 -a} = \frac{a -1}{a^2 - 2}$$
  \item $$\frac{x - 1}{x + 2 - \frac{x^2 + 2}{x - \frac{x - 2}{x + 1}}}$$
Comenzamos a expresar como una sola fracción a partir del denominador más embebido:
$$= \frac{x - 1}{x + 2 - \frac{x^2 + 2}{x - \frac{x - 2}{x + 1}}} = \frac{x - 1}{x + 2 - \frac{x^2 + 2}{\frac{x(x + 1) - x + 2}{x + 1}}} = \frac{x - 1}{x + 2 - \frac{x^2 + 2}{\frac{x^2 + x - x + 2}{x + 1}}}$$
Efectuamos operaciones:
$$= \frac{x - 1}{x + 2 - \frac{x^2 + 2}{\frac{x^2 + 2}{x + 1}}}$$
Reducimos a una sola división multiplicando extremos y medios:
$$= \frac{x - 1}{x + 2 - \frac{(x^2 + 2)(x + 1)}{x^2 + 2}}$$
Reducimos:
$$= \frac{x - 1}{x + 2 - x - 1} = x - 1$$



\end{enumerate}
\end{document}











































